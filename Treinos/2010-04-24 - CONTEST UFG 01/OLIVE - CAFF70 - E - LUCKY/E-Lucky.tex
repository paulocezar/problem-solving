% XeLaTeX can use any Mac OS X font. See the setromanfont command below.
% Input to XeLaTeX is full Unicode, so Unicode characters can be typed directly into the source.

% The next lines tell TeXShop to typeset with xelatex, and to open and save the source with Unicode encoding.

%!TEX TS-program = xelatex
%!TEX encoding = UTF-8 Unicode

\documentclass[14pt]{article}
\usepackage{geometry}                % See geometry.pdf to learn the layout options. There are lots.
\geometry{a4paper}                   % ... or a4paper or a5paper or ... 
%\geometry{landscape}                % Activate for for rotated page geometry
%\usepackage[parfill]{parskip}    % Activate to begin paragraphs with an empty line rather than an indent
\usepackage{graphicx}
\usepackage{amssymb}

% Will Robertson's fontspec.sty can be used to simplify font choices.
% To experiment, open /Applications/Font Book to examine the fonts provided on Mac OS X,
% and change "Hoefler Text" to any of these choices.

\usepackage{fontspec,xltxtra,xunicode}
\defaultfontfeatures{Mapping=tex-text}
\setromanfont[Mapping=tex-text]{Hoefler Text}
\setsansfont[Scale=MatchLowercase,Mapping=tex-text]{Gill Sans}
\setmonofont[Scale=MatchLowercase]{Andale Mono}

\title{Treino 1 - Maratona de Programa\c{c}\~ao }
\author{Universidade Federal de Goi\'as}
\date{24/04/2010}                                           % Activate to display a given date or no date

\begin{document}
\maketitle

% For many users, the previous commands will be enough.
% If you want to directly input Unicode, add an Input Menu or Keyboard to the menu bar 
% using the International Panel in System Preferences.
% Unicode must be typeset using a font containing the appropriate characters.
% Remove the comment signs below for examples.

% \newfontfamily{\A}{Geeza Pro}
% \newfontfamily{\H}[Scale=0.9]{Lucida Grande}
% \newfontfamily{\J}[Scale=0.85]{Osaka}

% Here are some multilingual Unicode fonts: this is Arabic text: {\A السلام عليكم}, this is Hebrew: {\H שלום}, 
% and here's some Japanese: {\J 今日は}.

\section{Problema E - Lucky Numbers}

\subsection{Descri\c{c}\~ao}
Como geralmente competi\c{c}\~oes de programa\c{c}\~ao tem problemas relacionados a deteminadas sequ\^encias, nesse treinamento n\~ao poderia ser diferente.
Definimos aqui uma sequ\^encia de n\'umeros que vamos chamar de \emph{ Lucky Numbers } ( N\'umeros da Sorte, pelo menos pro primeiro time que acertar ).

A sequ\^encia da sorte \'e a seque\^encia infinita de todos os inteiros, em ordem crescente, que podem ser representados como pot\^encias de 5 ( isto \'e: $5^{k}$ onde k \'e um inteiro positivo ) ou como uma soma de pot\^encias distintas de 5 ( ou seja, $5^{ a1} + 5^{ a2} + 5^{ a3}  + ... $, onde a1, a2, a3, ... s\~ao inteiros positivos distintos ). Todos os n\'umeros na sequ\^encia da sorte s\~ao chamados Lucky Numbers. Os primeiros Lucky Numbers s\~ao: $5, 25, 30, 125, 130, 150, ... $ 

\subsection{Tarefa}
Dado n, sua tarefa \'e encontrar o n-\'esimo Lucky Number.

\subsection{Entrada}
A primeira linha da entrada cont\'em um inteiro \textbf{t}, $ t \leq 1000 $,  o n\'umero de casos teste. Ent\~ao, seguem t linhas, cada uma contendo um inteiro \textbf{n}, $1 \leq n \leq 2^{21} $.

\subsection{Sa\'{\i}da}
Para cada caso de teste, imprima o n-\'esimo lucky number em uma \'unica linha. Nenhuma resposta excede o valor de $2^{64}$.

\subsection{Exemplo}
\subsubsection{Entrada}
\begin{verbatim}
4
1
2
3
9
 \end{verbatim}
\subsubsection{Sa\'{\i}da}
 \begin{verbatim}
5
25
30
630
 \end{verbatim}


\end{document}  