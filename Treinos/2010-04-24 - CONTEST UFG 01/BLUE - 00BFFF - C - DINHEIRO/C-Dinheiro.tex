% XeLaTeX can use any Mac OS X font. See the setromanfont command below.
% Input to XeLaTeX is full Unicode, so Unicode characters can be typed directly into the source.

% The next lines tell TeXShop to typeset with xelatex, and to open and save the source with Unicode encoding.

%!TEX TS-program = xelatex
%!TEX encoding = UTF-8 Unicode

\documentclass[14pt]{article}
\usepackage{geometry}                % See geometry.pdf to learn the layout options. There are lots.
\geometry{a4paper}                   % ... or a4paper or a5paper or ... 
%\geometry{landscape}                % Activate for for rotated page geometry
%\usepackage[parfill]{parskip}    % Activate to begin paragraphs with an empty line rather than an indent
\usepackage{graphicx}
\usepackage{amssymb}

% Will Robertson's fontspec.sty can be used to simplify font choices.
% To experiment, open /Applications/Font Book to examine the fonts provided on Mac OS X,
% and change "Hoefler Text" to any of these choices.

\usepackage{fontspec,xltxtra,xunicode}
\defaultfontfeatures{Mapping=tex-text}
\setromanfont[Mapping=tex-text]{Hoefler Text}
\setsansfont[Scale=MatchLowercase,Mapping=tex-text]{Gill Sans}
\setmonofont[Scale=MatchLowercase]{Andale Mono}

\title{Treino 1 - Maratona de Programa\c{c}\~ao }
\author{Universidade Federal de Goi\'as}
\date{24/04/2010}                                           % Activate to display a given date or no date

\begin{document}
\maketitle

% For many users, the previous commands will be enough.
% If you want to directly input Unicode, add an Input Menu or Keyboard to the menu bar 
% using the International Panel in System Preferences.
% Unicode must be typeset using a font containing the appropriate characters.
% Remove the comment signs below for examples.

% \newfontfamily{\A}{Geeza Pro}
% \newfontfamily{\H}[Scale=0.9]{Lucida Grande}
% \newfontfamily{\J}[Scale=0.85]{Osaka}

% Here are some multilingual Unicode fonts: this is Arabic text: {\A السلام عليكم}, this is Hebrew: {\H שלום}, 
% and here's some Japanese: {\J 今日は}.

\section{Problema C - Dinheiro importa}

\subsection{Descri\c{c}\~ao}


Nossa triste hist\'oria come\c{c}a com um grupo de grandes amigos. Juntos, eles fizeram uma viagem ao distante pa\'{\i}s da Quadrad\^onia. Durante sua estada no pa\'{\i}s, v\'arios eventos que s\~ao muito horr\'{\i}veis pra ser mencionados ocorreram. O resultado ? A \'ultima noite da viagem terminou com uma vasta troca de ``Nunca mais quero te ver de novo!''s. Um c\'alculo r\'apido indica que a frase pode ter sido dita quase 50 milh\~oes de vezes!

De volta ao Brasil, os integrantes do grupo de ex-amigos perceberam que n\~ao haviam dividido os custos da viagem de forma uniforme. Alguns ficaram devendo  v\'arios milhares, e regularizar as d\'{\i}vidas acabou se tornando um problema complexo, pois como muitos no grupo n\~ao querem nem falar com alguns dos ex-amigos, imagine dar dinheiro aos mesmos.

Naturalmente sempre h\'a alguem para ajudar, e nesse caso, a criatura com um bom cora\c{c}\~ao \'e voc\^e. Ap\'os ficar sabendo da trag\'edia, 
voc\^e perguntou a cada pessoa do grupo quanto ela ficou devendo, ou quanto est\~ao devendo pra ela, al\'em de, com quais pessoas do grupo ela contiua mantendo contato, e agora, com base nessas informa\c{c}\~oes, vai informar para o grupo se eles ser\~ao ou n\~ao capazes de quitar suas dividas.

\subsection{Tarefa}
Dadas as quantias que cada um pegou emprestado ou ficou devendo, sua tarefa \'e determinar se \'e poss\'{\i}vel ou n\~ao que as d\'{\i}vidas sejam quitadas sem que pessoas que n\~ao est\~ao mais conversando tenham quem interagir.

\subsection{Entrada}
A primeira linha cont\'em um inteiro T ( $0 \leqslant T \leqslant 30$\ ), especificando o n\'umero de casos de teste.

Os casos de teste tem a seguinte estrutura: a primeira linha de cada caso de teste cont\'em dois inteiros N ($\ 2 \leqslant N \leqslant 10000\ $), e M ($\ 0 \leqslant M \leqslant 50000\ $), o n\'umero de amigos, e de pares de pessoas que ainda mant\'em rela\c{c}\~oes de amizade. Ent\~ao, seguem N linhas, cada uma contendo um inteiro V (\ $-10000 \leqslant V \leqslant 10000\ $) indicando quanto cada pessoa deve receber ( ou pagar se $ V < 0\ $ ). A soma desses valores \'e zero. As pr\'oximas M linhas descrevem as amizades que realmente eram verdadeiras e permaneceram depois de todos os problemas. Cada amizade \'e descrita por dois inteiros X, e Y (\ $0 \leqslant x < y \leqslant N-1$\ ) indicando que as pessoas X e Y permanecem amigas.

\subsection{Sa\'{\i}da}
A sa\'{\i}da para cada caso teste ser\'a uma \'unica linha contendo ``POSSIVEL" ou ``IMPOSSIVEL".

\subsection{Exemplo}
\subsubsection{Entrada}
\begin{verbatim}
2
5 3
100
-75
-25
-42
42
0 1
1 2
3 4
4 2
15
20
-10
-25
0 2
1 3
 \end{verbatim}
\subsubsection{Sa\'{\i}da}
 \begin{verbatim}
 POSSIVEL
 IMPOSSIVEL
 \end{verbatim}


\end{document}  