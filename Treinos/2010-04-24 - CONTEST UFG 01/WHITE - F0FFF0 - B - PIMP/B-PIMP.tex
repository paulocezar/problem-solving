% XeLaTeX can use any Mac OS X font. See the setromanfont command below.
% Input to XeLaTeX is full Unicode, so Unicode characters can be typed directly into the source.

% The next lines tell TeXShop to typeset with xelatex, and to open and save the source with Unicode encoding.

%!TEX TS-program = xelatex
%!TEX encoding = UTF-8 Unicode

\documentclass[14pt]{article}
\usepackage{geometry}                % See geometry.pdf to learn the layout options. There are lots.
\geometry{a4paper}                   % ... or a4paper or a5paper or ... 
%\geometry{landscape}                % Activate for for rotated page geometry
%\usepackage[parfill]{parskip}    % Activate to begin paragraphs with an empty line rather than an indent
\usepackage{graphicx}
\usepackage{amssymb}

% Will Robertson's fontspec.sty can be used to simplify font choices.
% To experiment, open /Applications/Font Book to examine the fonts provided on Mac OS X,
% and change "Hoefler Text" to any of these choices.

\usepackage{fontspec,xltxtra,xunicode}
\defaultfontfeatures{Mapping=tex-text}
\setromanfont[Mapping=tex-text]{Hoefler Text}
\setsansfont[Scale=MatchLowercase,Mapping=tex-text]{Gill Sans}
\setmonofont[Scale=MatchLowercase]{Andale Mono}

\title{Treino 1 - Maratona de Programa\c{c}\~ao }
\author{Universidade Federal de Goi\'as}
\date{24/04/2010}                                           % Activate to display a given date or no date

\begin{document}
\maketitle

% For many users, the previous commands will be enough.
% If you want to directly input Unicode, add an Input Menu or Keyboard to the menu bar 
% using the International Panel in System Preferences.
% Unicode must be typeset using a font containing the appropriate characters.
% Remove the comment signs below for examples.

% \newfontfamily{\A}{Geeza Pro}
% \newfontfamily{\H}[Scale=0.9]{Lucida Grande}
% \newfontfamily{\J}[Scale=0.85]{Osaka}

% Here are some multilingual Unicode fonts: this is Arabic text: {\A السلام عليكم}, this is Hebrew: {\H שלום}, 
% and here's some Japanese: {\J 今日は}.

\section{Problema B - PIMP My Ride}

\subsection{Descri\c{c}\~ao}
Hoje, existem muitos carros, motos, caminh\~oes e outros ve\'iculos l\'a fora, nas ruas que precisam seriamente de algumas remodela\c{c}\~oes.
Para a sorte desses ve\'iculos ( ou de seus donos ), um programa na TV tem tomado conta desses velhos calhambeques, selecionando periodicamente alguns felizardos para participarem de um quadro onde sua lata velha \'e completamente reformada.
Os v\'arios servi\c{c}os que devem ser feitos no carro, como pintura, decora\c{c}\~ao de interior, alinhamento, montagem de som, etc; s\~ao realizados em diferentes garagens. E mais, quanto melhor a apar\^encia do carro, mais caro eles costumam cobrar. Por exemplo, um pintor geralmente cobra mais quando vai pintar um carro com o interior todo de couro. Todas essas sobretaxas, dependen de qual servi\c{c}o est\'a sendo feito e de quais servi\c{c}os foram feitos antes.
Para saber quais carros escolher para participar do programa, a emissora prefere os que tenham um custo menor para realizar todos os servi\c{c}os necess\'arios, e para isso precisa da sua ajuda.

\subsection{Tarefa}
Servi\c{c}os individuais s\~ao numerados de $1$ a N. Dados o pre\c{c}o base p para cada servi\c{c}o e a sobretaxa s ( em R\$ ) para todo par de servi\c{c}os ( i, j ) com $i \neq j$, significando que voc\^e deve pagar s R\$ adicionais pelo servi\c{c}o i, se e somente se o servi\c{c}o j tiver sido realizado antes, voc\^e deve calcular o custo m\'inimo total necess\'ario para \emph{ PIMPAR } o carro por completo.

\subsection{Entrada}
A primeira linha cont\'em o n\'umero de casos teste. Cada caso teste \'e composto de um inteiro N, $1 \leqslant n \leqslant 14$; o n\'umero de servi\c{c}os. Em seguida, N linhas, cada uma contendo exatamente N inteiros. A i-\'esimo linha cont\'em as sobretaxas que devem ser pagas na garagem i para realizar o i-\'esimo servi\c{c}o e o valor base para realizar o i-\'esimo servi\c{c}o. Mais precisamente, na i-\'esima linha, o i-\'esimo inteiro \'e o pre\c{c}o base para realizar o servi\c{c}o i e o j-\'esimo inteiro (\ $j \neq i$\ ) \'e a sobretaxa para o servi\c{c}o i que \'e cobrada se o servi\c{c}o j foi realizado antes. Os valores ser\~ao inteiros n\~ao negativos menores ou iguais a $100000$.


\subsection{Sa\'{\i}da}
A sa\'ida de cada caso teste \'e composta por $3$ linhas.
A primeira linha cont\'em  ``Teste \#i:" onde i e o n\'umero do caso de teste come\c{c}ando de $1$.
Uma \'unica linha contendo a frase: “Voce foi oficialmente pimpado por apenas R\$ p.", onde p \'e o pre\c{c}o m\'inimo total.
E por \'ultimo, uma linha em branco finaliza a sa\'ida.

\subsection{Exemplo}
\subsubsection{Entrada}
\begin{verbatim}
2
2
10 10
9000 10
3
14 23 0
0 14 0
1000 9500 14
 \end{verbatim}
\subsubsection{Sa\'{\i}da}
 \begin{verbatim}
Teste #1:
Voce foi oficialmente pimpado por apenas R$ 30.

Teste #2:
Voce foi oficialmente pimpado por apenas R$ 42.

 \end{verbatim}


\end{document}  